\documentclass[11pt]{report}

% ---------------------- Pakete ----------------------
\usepackage[utf8]{inputenc}
\usepackage[T1]{fontenc}
\usepackage[ngerman]{babel}
\usepackage[a4paper, margin=2cm]{geometry}
\usepackage{setspace}         % 1.5-zeilig
\usepackage{titlesec}         % Überschriftendesign
\usepackage{fancyhdr}         % Kopf-/Fußzeile
\usepackage{graphicx}         % Bilder
\usepackage{booktabs}         % Tabellen
\usepackage{caption, subcaption}
\usepackage{biblatex}         % Literatur
\usepackage{hyperref}         % klickbare Verweise
\usepackage{tocloft}          % Inhaltsverzeichnis-Anpassung
\usepackage{etoolbox}         % Patch für Kapitel-Kopfzeilen

\addbibresource{literatur/quellen.bib}  % Bib-Datei einbinden

% ---------------------- Formatierung ----------------------
\onehalfspacing

\titleformat{\chapter}[hang]{\normalfont\huge\bfseries}{\thechapter\quad}{0pt}{}
\titlespacing*{\chapter}{0pt}{12pt}{20pt}

\titleformat{\section}{\normalfont\Large\bfseries}{\thesection\quad}{0pt}{}
\titlespacing*{\section}{0pt}{8pt}{15pt}

\titleformat{\subsection}{\normalfont\large\bfseries}{\thesubsection\quad}{0pt}{}
\titlespacing*{\subsection}{0pt}{4pt}{10pt}

% ---------------------- Kopf-/Fußzeile ----------------------
\pagestyle{fancy}
\fancyhf{}
\fancyhead[L]{Mai-Anh Pham}
\fancyhead[R]{\leftmark}
\renewcommand{\headrulewidth}{0.4pt}
\renewcommand{\footrulewidth}{0pt}

% Kapitelbeginn = fancy
\patchcmd{\chapter}{\thispagestyle{plain}}{\thispagestyle{fancy}}{}{}

% ---------------------- TOC: Einträge ohne Seitenzahl ----------------------
\newcommand{\nocountchapter}[1]{
  \chapter*{#1}
  \addcontentsline{toc}{chapter}{#1}
  \markboth{#1}{}
  \thispagestyle{empty}
}

% ---------------------- Anhang Setup ----------------------
\usepackage{chngcntr} % Abschnittszählung für Anhang
\newcommand{\anhangchapter}[1]{%
  \refstepcounter{chapter}%
  \chapter*{Anhang~\Alph{chapter}:~#1}
  \addcontentsline{toc}{chapter}{Anhang~\Alph{chapter}:~#1}%
  \markboth{Anhang~\Alph{chapter}:~#1}{}
  \setcounter{section}{0}
  \renewcommand{\thesection}{\Alph{chapter}.\arabic{section}}%
}

% ---------------------- Dokument ----------------------
\begin{document}

% Titelblatt
\begin{titlepage}
    \thispagestyle{empty}
  
    \begin{center}
      \vspace*{-1cm}
      \includegraphics[height=2.5cm]{bilder/DB_logo_red_filled_1000px_rgb.png}
      \hfill
      \includegraphics[height=2.5cm]{bilder/DHBW_MA_Logo.jpg}
  
      \vspace{1.5cm}
  
      {\Large\bfseries
      Duale Hochschule Baden-Württemberg\\
      Mannheim}
  
      \vspace{1cm}
  
      {\large\bfseries Bachelorarbeit}
  
      \vspace{1cm}
  
      {\LARGE\bfseries
      Titel
      }
  
      \vspace{1cm}
  
      {\large
      Studiengang Wirtschaftsinformatik – Sales \& Consulting\\
      Bearbeitungszeitraum: 
      }
  
      \vfill
  
      \begin{tabular}{@{}ll}
      \textbf{Verfasser:} & Mai-Anh Pham \\
      \textbf{Matrikelnummer:} & 5621973 \\
      \textbf{Kurs:} & WWI22SCB \\
      \textbf{Studiengangsleiter:} & Prof. Dr.-Ing. Clemens Martin \\
      \end{tabular}
  
      \vspace{1cm}
  
      \begin{tabular}{@{}ll}
      \textbf{Wiss. Betreuer:} & Prof. Dr.-Ing. Clemens Martin \\
      \textbf{Mailadresse:} & martin@dhbw-mannheim.de \\
      \end{tabular}
  
      \vspace{1cm}
  
      \begin{tabular}{@{}ll}
      \textbf{Ausbildungsbetrieb:} & DB Systel GmbH \\
      & Jürgen-Ponto-Platz 1 \\
      & 60239 Frankfurt am Main / Deutschland \\
      \end{tabular}
  
      \vspace{1cm}
  
      \begin{tabular}{@{}ll}
      \textbf{Unternehmensbetreuer:} & XYZ \\
      \textbf{Telefon (Firma):} & +49 \\
      \textbf{Mailadresse (Firma):} & XYZ \\
      \end{tabular}
  
    \end{center}
  
  \end{titlepage}
  
\cleardoublepage

% Römische Seitenzahlen (groß)
\pagenumbering{Roman}

% Abstract (ohne Seitenzahl, aber im TOC)
\nocountchapter{Abstract}
\input{abstract}

% Danksagung (dito)
\nocountchapter{Danksagung}
\input{danksagung}

% Inhaltsverzeichnis etc.
\tableofcontents
\listoffigures
\listoftables
\cleardoublepage

% Arabische Seitenzahlen starten mit Kapitel 1
\pagenumbering{arabic}

% Kapitel
\input{kapitel/einleitung}
\input{kapitel/methodik}
\input{kapitel/fazit}

% Literaturverzeichnis
\cleardoublepage
\pagenumbering{roman}
\setcounter{page}{1}
\printbibliography[heading=bibintoc]

% Anhangsverzeichnis (optional manuell)
\cleardoublepage
\chapter*{Anhangsverzeichnis}
\addcontentsline{toc}{chapter}{Anhangsverzeichnis}

% Anhang (A, B, ... mit Unterabschnitten wie A.1)
\anhangchapter{Versuchsaufbau}
Text des ersten Anhangs...

\anhangchapter{Weitere Tabellen}
Noch mehr...

% Ehrenwörtliche Erklärung (ohne Seitenzahl, aber im TOC)
\cleardoublepage
\nocountchapter{Ehrenwörtliche Erklärung}

\begin{spacing}{1.5}
    Hiermit erkläre ich, dass ich die vorliegende Arbeit selbstständig und ohne fremde Hilfe  
    verfasst und keine anderen Hilfsmittel als die angegebenen verwendet habe.  
    
    Insbesondere versichere ich, dass ich alle wörtlichen und sinngemäßen Übernahmen aus  
    anderen Werken als solche kenntlich gemacht habe.  
    \end{spacing}
    
    \vspace{2.5cm}
    
    \noindent
    \begin{minipage}[t]{0.45\textwidth}
      \centering
      \textbf{Ort, Datum}
    \end{minipage}
    \hfill
    \begin{minipage}[t]{0.45\textwidth}
      \centering
      \textbf{Unterschrift}
    \end{minipage}
    


\end{document}
